\section{Boolean algebras}
\label{sec:boolean_algebras}

Next we investigate Boolean algebras
in terms of algebraic structures.
The resulting correspondence between Boolean algebras and Boolean rings
is motivated by Exercise~2 in the book
\textquote{Introduction to Boolean Algebras}
\cite[p.~18]{halmos:boolean_algebras}.
Afterwards we give some examples for Boolean algebras
and close this section by examining ultrafilters on Boolean algebras,
which are the main object of interest
in Section \ref{sec:representation_theorems}.

\medskip

In contrast to the approach in Section \ref{sec:lattices} it is more common
to directly formulate the axioms for Boolean algebras in the language
$\symcal{L} \equal \lbrace \vee \mathcomma \wedge \mathcomma
\, \prime \mathcomma 0 \mathcomma 1 \rbrace$.
This is convenient in that it also defines homomorphisms and substructures,
which are again Boolean algebras.
Usually, one takes the axioms listed below or their universal closures.

\begin{table}[h]
\centering
  \begin{tabular}{rccl}
    \textcolor{gray}{\texttt{1.}}
    &
    $\lparen a \vee b \rparen \vee c
    \equal a \vee \lparen b \vee c \rparen$
    &
    $\lparen a \wedge b \rparen \wedge c
    \equal a \wedge \lparen b \wedge c \rparen$
    &
    \phantom{\texttt{1.}}
    \\
    \textcolor{gray}{\texttt{2.}}
    &
    $a \vee b \equal b \vee a$
    &
    $a \wedge b \equal b \wedge a$
    &
    \phantom{\texttt{2.}}
    \\
    \textcolor{gray}{\texttt{3.}}
    &
    $a \vee \lparen a \wedge b \rparen \equal a$
    &
    $a \wedge \lparen a \vee b \rparen \equal a$
    &
    \phantom{\texttt{3.}}
    \\
    \textcolor{gray}{\texttt{4.}}
    &
    $a \vee a \equal a$
    &
    $a \wedge a \equal a$
    &
    \phantom{\texttt{4.}}
    \\
    \textcolor{gray}{\texttt{5.}}
    &
    $a \vee 0 \equal a$
    &
    $a \wedge 1 \equal a$
    &
    \phantom{\texttt{5.}}
    \\
    \textcolor{gray}{\texttt{6.}}
    &
    $a \vee 1 \equal 1$
    &
    $a \wedge 0 \equal 0$
    &
    \phantom{\texttt{6.}}
    \\
    \textcolor{gray}{\texttt{7.}}
    &
    $a \vee \lparen b \wedge c \rparen
      \equal
      \lparen a \vee b \rparen \wedge \lparen a \vee c \rparen$
    &
    $a \wedge \lparen b \vee c \rparen
      \equal
      \lparen a \wedge b \rparen \vee \lparen a \wedge c \rparen$
    &
    \phantom{\texttt{7.}}
    \\
    \textcolor{gray}{\texttt{8.}}
    &
    $a \vee a \prime \equal 1$
    &
    $a \wedge a \prime \equal 0$
    &
    \phantom{\texttt{8.}}
    \\
    \textcolor{gray}{\texttt{9.}}
    &
    $1 \prime \equal 0$
    &
    $0 \prime \equal 1$
    &
    \phantom{\texttt{9.}}
    \\
    \textcolor{gray}{\texttt{10.}}
    &
    \multicolumn{2}{c}{$a \equal \lparen a \prime \rparen \prime$}
    &
    \phantom{\texttt{10.}}
    \\
    \textcolor{gray}{\texttt{11.}}
    &
    $\lparen a \vee b \rparen \prime \equal a \prime \wedge b \prime$
    &
    $\lparen a \wedge b \rparen \prime \equal a \prime \vee b \prime$
    &
    \phantom{\texttt{11.}}
    \\
  \end{tabular}
\end{table}

It is easy to see how this list of axioms connects
to the lattice-based point of view.
For example, Lines 1 to 3 are the definition of lattices
and Lines 5 and 6 are the definition of complements in bounded lattices.
However, this list of axioms is not minimal;
for instance we showed that Lines 4 and 10
are rather consequences of the other axioms.

Huntington proved \cite{huntingtion:axiomset_boolean}
that the commutative laws (Line 2), the identity laws (Line~5),
the distributive laws (Line 7), and the complement laws (Line 8)
form a minimal axiom set for Boolean algebras.

\medskip

We now develop another point of view for Boolean algebras
based on classical algebra.
For this purpose we introduce \emph{Boolean rings}
and show how every Boolean algebra
gives rise to such a ring.

\medskip

\begin{definition}
  A unitary ring $R$ is called a \emph{Boolean ring}
  if all of its elements are idempotent
  (that is $a^2 \equal a$ for all $a \in R$).
\end{definition}

\begin{proposition}
\label{prop:boolean_ring}
  A Boolean ring is commutative and of characteristic 2.
\end{proposition}

\begin{proof}
  Let $R$ be a Boolean ring and $a \mathcomma b \in R$.
  Then
  \[
    a \mathplus b
    \equal
    \lparen a \mathplus b \rparen ^2
    \equal
    a ^2 \mathplus ab \mathplus ba \mathplus b ^2
    \equal
    a \mathplus ab \mathplus ba \mathplus b,
  \]
  so $ab \mathplus ba \equal 0$.
  Consequently the choice $a \equal b$ leads to
  \[
    0
    \equal
    a ^2 \mathplus a ^2
    \equal
    a \mathplus a,
  \]
  hence $R$ has characteristic 2.
  In particular both $ab$ and $ba$ are negatives of $ab$
  and we obtain $ab \equal ba$.
\end{proof}

\begin{theorem}
\label{thm:algebra_ring}
  Let
  $\lparen B \mathcomma \vee \mathcomma \wedge \mathcomma \, \prime
  \mathcomma 0 \mathcomma 1 \rparen$
  be a Boolean algebra.
  Then
  $\lparen B \mathcomma \mathplus \mathcomma \minus \mathcomma \cdotp
  \mathcomma 0 \mathcomma 1 \rparen$
  is a Boolean ring,
  where
  \begin{align*}
    a \mathplus b
    &\coloneq
    \lparen a \wedge b \prime \rparen \vee \lparen a \prime \wedge b \rparen\\
    \minus a
    &\coloneq
    a\\
    a \cdotp b
    &\coloneq
    a \wedge b
  \end{align*}
  for $a \mathcomma b \in B$.
\end{theorem}

\begin{proof}
  Once the above operations are established
  it is merely a matter of routine to check that these make $B$ a Boolean ring.
  For example the commutativity of the join and the meet leads to
  \[
    a \mathplus b
    \equal
    \lparen a \wedge b \prime \rparen \vee \lparen a \prime \wedge b \rparen
    \equal
    \lparen b \wedge a \prime \rparen \vee \lparen b \prime \wedge a \rparen
    \equal
    b \mathplus a
  \]
  for all $a \mathcomma b \in B$,
  whence the arising addition is also commutative.
\end{proof}

\medskip

In Section \ref{sec:lattices} we have seen the close relationship
between lattices
and ordered sets in which infima and suprema exist for all finite subsets.
The next result reveals a similar connection
between Boolean rings and Boolean algebras.
It is easy to see that the constructions
from Theorems \ref{thm:algebra_ring} and \ref{thm:ring_algebra}
are the counterpart of each other,
such that we have a one-to-one correspondence between both structures.

\medskip

\begin{theorem}
\label{thm:ring_algebra}
  Let
  $\lparen R \mathcomma \mathplus \mathcomma \minus \mathcomma \cdotp
  \mathcomma 0 \mathcomma 1 \rparen$
  be a Boolean ring.
  Then
  $\lparen R \mathcomma \vee \mathcomma \wedge \mathcomma \, \prime
  \mathcomma 0 \mathcomma 1 \rparen$
  is a Boolean algebra,
  where
  \begin{align*}
    a \vee b
    &\coloneq
    a \mathplus b \mathplus ab \\
    a \wedge b
    &\coloneq
    ab \\
    a \prime
    &\coloneq
    a \mathplus 1
  \end{align*}
  for $a \mathcomma b \in R$.
\end{theorem}

\begin{proof}
  Similarly to the proof of Theorem \ref{thm:algebra_ring},
  the verification of the lattice axioms is just an arithmetical issue.
  For instance, with Proposition \ref{prop:boolean_ring} one directly obtains
  \[
    a \vee \lparen a \wedge b \rparen
    \equal
    a \mathplus ab \mathplus a^2 b
    \equal
    a \mathplus ab \mathplus ab
    \equal
    a
  \]
  as well as
  \[
    a \wedge \lparen a \vee b \rparen
    \equal
    a \lparen a \mathplus b \mathplus ab \rparen
    \equal
    a^2 \mathplus ab \mathplus a^2b
    \equal
    a \mathplus ab \mathplus ab
    \equal
    a
  \]
  for all $a \mathcomma b \in R$,
  so that $\vee$ and $\wedge$ fulfill the absorption laws.
\end{proof}

\medskip

Next, we see some examples for Boolean algebras.

\medskip

\begin{example}
\label{ex:2}
  The set
  $\lbrace 0 \mathcomma 1 \rbrace$
  gives rise to the smallest non-degenerate Boolean algebra.
  A suitable choice of the lattice operations
  as well as the corresponding ring operations are stated in the tables below.

  \begin{table}[h]
  \centering
    \captionbox*{The lattice operations.}{%
      \begin{tabular}{c|cc}
        $\vee$ & $0$ & $1$ \\ \hline
        $0$    & $0$ & $1$ \\
        $1$    & $1$ & $1$
      \end{tabular}
      \quad
      \begin{tabular}{c|cc}
        $\wedge$ & $0$ & $1$ \\ \hline
        $0$      & $0$ & $0$ \\
        $1$      & $0$ & $1$
      \end{tabular}
    }
    \quad
    \quad
    \quad
    \captionbox*{The ring operations.}{%
      \begin{tabular}{c|cc}
        $\mathplus$ & $0$ & $1$ \\ \hline
        $0$         & $0$ & $1$ \\
        $1$         & $1$ & $0$
      \end{tabular}
      \quad
      \begin{tabular}{c|cc}
        $\cdotp$ & $0$ & $1$ \\ \hline
        $0$      & $0$ & $0$ \\
        $1$      & $0$ & $1$
      \end{tabular}
    }
  \end{table}
\end{example}

\begin{example}
\label{ex:powerset}
  Any set $X$ gives rise to a Boolean algebra
  $\left \lparen \Po{X} \mathcomma \cup \mathcomma \cap \mathcomma
  \, ^ \complement \mathcomma \varnothing \mathcomma X \right \rparen$.
  The addition in the corresponding Boolean ring is given by
  \[
    A \bigtriangleup B
    \equal
    \lparen A \setminus B \rparen \cup \lparen B \setminus A \rparen
  \]
  for $A \mathcomma B \subseteq X$
  and often called \emph{symmetric difference}.

  If $X \equal \varnothing$
  the above construction leads to the degenerate Boolean algebra,
  and if $X$ consists of exactly one element,
  the resulting Boolean algebra is isomorphic to Example~\ref{ex:2}.
\end{example}

\begin{example}
\label{ex:clopen}
  Let
  $\lparen X \mathcomma \symcal{O} \rparen$
  be a topological space.
  The subsets of $X$ that are both open and closed form a Boolean algebra.
\end{example}

\begin{example}
\label{ex:countable}
  The set of finite and cofinite subsets of $\N_0$ is a Boolean algebra.
\end{example}

\begin{remark}
  Let $\symcal{F}$ be the set of finite subsets of $\N_0$ and consider the map
  \[
    f \mathcolon \symcal{F} \rightarrow \N_0 \mathcomma \
    F \mapsto \sum_{n \in F} 2^n.
  \]
  Applying appropriate number theoretic methods,
  it is shown that f is a bijection.
  More precisely, one uses the same techniques as for deriving
  the unique representation of a natural number $n$
  in the base-$2$ numeral system,
  and $f^{\minus 1} \lparen n \rparen$ indicates the positions
  where the digit 1 is located in the binary representation of $n$.
  The German-speaking reader can read about this in more detail
  in Chapter 4 of the book
  \textquote{Elementare Zahlentheorie} by Remmert and Ullrich
  \cite{remmert:zahlentheorie-3}.

  Notably this result shows that $\symcal{F}$ is countable,
  and by taking complements we also see that
  the set of cofinite subsets of $\N_0$ is.
  In particular, Example \ref{ex:countable} is a countable Boolean algebra.
\end{remark}

\medskip

The final plan for this section is to work out
the definition and properties of ultrafilters on Boolean algebras.
The most general notion of filters is found in order theory
as shown in Section \ref{sec:orders}.
On Boolean algebras, a more concrete but equivalent definition
can be given using the meet.

\medskip

\begin{lemma}
\label{la:order_complement}
  Let
  $\lparen B \mathcomma \vee \mathcomma \wedge \mathcomma \, \prime
  \mathcomma 0 \mathcomma 1 \rparen$
  be a Boolean algebra.
  Then
  \[
    b \wedge c \equal 0
    \Leftrightarrow
    b \leq c \prime
  \]
  for all $b \mathcomma c \in B$.
\end{lemma}

\begin{proof}
  Let $b \mathcomma c \in B$.

  \fbox{$\Rightarrow$}
  It is
  \[
    b \wedge c \prime
    \equal
    \lparen b \wedge c \prime \rparen \vee 0
    \equal
    \lparen b \wedge c \prime \rparen \vee \lparen b \wedge c \rparen
    \equal
    b \wedge \lparen c \wedge c \prime \rparen
    \equal
    b \wedge 1
    \equal
    b,
  \]
  hence $b \leq c \prime$.

  \fbox{$\Leftarrow$}
  By definition of the order the assumption means
  $b \wedge c \prime \equal b$,
  so that
  \[
    b \wedge c
    \equal
    \lparen b \wedge c \prime \rparen \wedge c
    \equal
    b \wedge \lparen c \prime \wedge c \rparen
    \equal
    b \wedge 0
    \equal
    0.
  \]
\end{proof}

\begin{definition}
  Let
  $\lparen B \mathcomma \vee \mathcomma \wedge \mathcomma \, \prime
  \mathcomma 0 \mathcomma 1 \rparen$
  be a Boolean algebra.
  A subset $F \subseteq B$ is called a \emph{filter} on the Boolean algebra,
  if
  \begin{enumerate}
    \item $F \neq \varnothing$,
    \item $f \leq b$ implies $b \in F$ for all $f \in F$ and $b \in B$,
    \item $f \wedge g \in F$ for all $f \mathcomma g \in F$.
  \end{enumerate}
  A filter is called \emph{proper} if $F \ne B$.
\end{definition}

\begin{remark}
  A filter $F$ on a Boolean algebra is proper if and only if $0 \notin F$.
\end{remark}

\pagebreak

\begin{definition}
  The set of proper filters on $B$ is partially ordered by set inclusion.
  A maximal element in this set is called an \emph{ultrafilter}.
\end{definition}

\medskip

Although the definition of ultrafilters on general ordered sets
and on those arising from a Boolean algebra is the same,
ultrafilters on Boolean algebras have a very convenient characterization.
It is also practical that certain sets can always be extended to ultrafilters,
resulting in the fact
that on every non-degenerate Boolean algebra there exists
at least one ultrafilter.

\begin{proposition}
\label{prop:ultrafilter}
  Let
  $\lparen B \mathcomma \vee \mathcomma \wedge \mathcomma \, \prime
  \mathcomma 0 \mathcomma 1 \rparen$
  be a Boolean algebra and $U$ be a filter.
  The following are equivalent:
  \begin{enumroman}
    \item $U$ is an ultrafilter.
    \item For any $b \in B$ either $b \in U$ or $b \prime \in U$.
  \end{enumroman}
\end{proposition}

\begin{proof}
  \fbox{$\Rightarrow$}
  Let $U$ be an ultrafilter and $b \in B$.
  It is clear that not both $b$ and $b \prime$ are contained in $U$,
  for otherwise also $0 \equal b \wedge b \prime \in U$,
  contradicting that $U$ is a proper filter.
  Without loss of generality we may assume $b \prime \notin U$
  and show that in this case $b \in U$.
  To this end, consider
  \[
    V
    \coloneq
    \lbrace
    c \in B \pipe \exists u \in U \text{ with } u \wedge b \leq c
    \rbrace.
  \]

  From $U \ne \varnothing$ one obtains $V \ne \varnothing$,
  and since $\leq$ is transitive $V$ is closed upwards.
  For~$v_1 \mathcomma v_2 \in V$ there are $u_1 \mathcomma u_2 \in U$
  with $u_1 \wedge b \leq v_1$ and $u_2 \wedge b \leq v_2$.
  We see that
  \[
    \underbrace{ \lparen u_1 \wedge u_2 \rparen }_{ \in U } \wedge \: b
    \equal
    \lparen u_1 \wedge b \rparen \wedge \lparen u_2 \wedge b \rparen
    \leq
    v_1 \wedge v_2,
  \]
  so $v_1 \wedge v_2 \in V$.

  Note that $u \wedge b \ne 0$ for all $u \in U$,
  since by Lemma \ref{la:order_complement}
  \[
    u \wedge b \equal 0
    \Rightarrow
    u \leq b \prime
    \Rightarrow
    b \prime \in U.
  \]
  Consequently $0 \notin V$,
  so overall $V$ is a proper filter.

  Finally we clearly have that $b \in V$ and $U \subseteq V$.
  Since $U$ is a maximal proper filter,
  the above observation leads to $U \equal V$,
  hence $b \in U$.

  \fbox{$\Leftarrow$}
  Let $U$ be a filter fulfilling (ii).
  Clearly $U$ is a proper filter and it is left to show maximality.
  To this end, consider a proper filter $V$ with $U \subseteq V$
  and assume there exists some $b \in V \setminus U$.
  Then $b \prime \in U$ by assumption (ii),
  and so also $b \prime \in V$.
  Since $b \mathcomma b \prime \in V$
  we have $0 \equal b \wedge b \prime \in V$,
  contradicting that $V$ is a proper filter.
  Thus, $V \equal U$ and $U$ is maximal in the set of proper filters.
\end{proof}

\begin{proposition}
\label{prop:intersectionstable}
  Let
  $\lparen B \mathcomma \vee \mathcomma \wedge \mathcomma \, \prime
  \mathcomma 0 \mathcomma 1 \rparen$
  be a Boolean algebra and $D \subset B$ such that
  \begin{itemize}
    \item $0 \notin D$,
    \item $d \wedge e \in D$ for all $d \mathcomma e \in D$.
  \end{itemize}
  Then there exists an ultrafilter $U$ with $D \subseteq U$.
\end{proposition}

\begin{proof}
  Similarly to the proof of Proposition \ref{prop:ultrafilter} we see that
  \[
    F \coloneq
    \lbrace
    b \in B
    \pipe
    \exists d \in D \text{ with } d \leq b
    \rbrace
  \]
  is a proper filter with $D \subseteq F$.
  Thus, the set
  \[
    \mupGamma
    \coloneq
    \lbrace G \subset B
    \pipe
    G \text{ is a proper filter with } D \subseteq G
    \rbrace
  \]
  is nonempty.
  Clearly for any chain $\symcal{C} \subseteq \mupGamma$
  the set $\bigcup \symcal{C}$
  is an upper bound for $\symcal{C}$ in $\mupGamma$.
  So by Zorn's lemma (cf.\ Theorem \ref{thm:zorn}),
  there is a maximal element $U$ in $\mupGamma$.
  This $U$ is an ultrafilter,
  for if $V$ is any proper filter with $U \subseteq V$,
  then $D \subseteq V$ and so $V \in \mupGamma$.
\end{proof}

\begin{corollary}
  Let
  $\lparen B \mathcomma \vee \mathcomma \wedge \mathcomma \, \prime
  \mathcomma 0 \mathcomma 1 \rparen$
  be a Boolean algebra and $D \subset B$.
  If
  $
    \bigwedge_{ d \in D_0 } d \ne 0
  $
  for any finite subset $D_0 \subseteq D$,
  then there exists an ultrafilter $U$ with $D \subseteq U$.
\end{corollary}

\begin{proof}
  Apply Proposition \ref{prop:intersectionstable} to the set
  $
    D \cup
    \lbrace
    \bigwedge_{ d \in D_0 } d
    \pipe
    D_0 \subseteq D \text{ finite}
    \rbrace
  $.
\end{proof}

\begin{remark}
  The application of the preceding corollary
  to the set $\lbrace 1 \rbrace$ shows
  that every non-degenerate Boolean algebra has at least one ultrafilter.
\end{remark}
