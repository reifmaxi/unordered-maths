\section{Representation theorems for Boolean algebras}
\label{sec:representation_theorems}

A natural question that arises in the study of a theory is
whether there is a common form of representation for all models.
Example \ref{ex:powerset} gives a first hint to approach this problem.
However, Example \ref{ex:countable} shows
that this approach is not enough for cardinality reasons
(cf.~the corresponding remark).
The most satisfying answer to this concern is given by Theorem \ref{thm:stone1},
and its topological version Theorem \ref{thm:stone2}.
The presentation chosen here is the formal elaboration
of a remark from the book \textquote{Introduction to Boolean Algebras}
by Steven Gavant and Paul Halmos \cite[p.~190]{halmos:boolean_algebras}.

\medskip

\begin{theorem}
\label{thm:stone1}
  For any Boolean algebra
  $\lparen B \mathcomma \vee \mathcomma \wedge \mathcomma \, \prime
  \mathcomma 0 \mathcomma 1 \rparen$
  there exists a set $X$ such that~$B$ can be embedded into $\Po{X}$.
\end{theorem}

\begin{proof}
  Let $\symcal{U}$ be the set of ultrafilters on $B$ and
  \[
    f
    \! \mathcolon
    B
    \rightarrow
    \Po{ \symcal{U} }
    \mathcomma \
    b
    \mapsto
    \lbrace
      U \in \symcal{U}
    \pipe
      b \in U
    \rbrace.
  \]
  We immediately see that
  $f \lparen 0 \rparen \equal \varnothing$
  since ultrafilters are proper filters,
  and $f \lparen 1 \rparen \equal \symcal{U}$
  since filters are nonempty and closed upwards.
  Let $b \mathcomma c \in B$.
  Proposition \ref{prop:ultrafilter} yields
  that~$f \lparen b \prime \rparen
  \equal \symcal{U} \setminus f \lparen b \rparen$,
  and
  $f \lparen b \wedge c \rparen
  \equal f \lparen b \rparen \cap f \lparen c \rparen$
  follows directly from the definition of filters.
  Since
  \[
    b \mathcomma c \notin U
    \Leftrightarrow
    b \prime \mathcomma c \prime \in U
    \Leftrightarrow
    b \prime \wedge c \prime \in U
    \Leftrightarrow
    \lparen b \vee c \rparen \prime \in U
    \Leftrightarrow
    b \vee c \notin U
  \]
  for any ultrafilter $U$ we additionally obtain that
  $f \lparen b \vee c \rparen
  \equal f \lparen b \rparen \cup f \lparen c \rparen$,
  hence $f$ is a homomorphism for Boolean algebras
  in the sense described at the beginning of
  Section~\ref{sec:boolean_algebras}.

  Let $d \mathcomma e \in B$ with $d \ne e$.
  The order $\leq$ is antisymmetric,
  so we have $d \nleq e$ or $e \nleq d$;
  without loss of generality we assume $d \nleq e$.
  By Lemma \ref{la:order_complement} this means
  $d \wedge e \prime \ne 0$,
  so due to Proposition \ref{prop:intersectionstable}
  there is an ultrafilter $U$ with
  $\lbrace d \wedge e \prime \rbrace \subseteq U$.
  Clearly $U \in f \lparen d \rparen$
  but~$U \notin f \lparen e \rparen$,
  thus $f$ is injective.
\end{proof}

\begin{theorem}[Stone's Representation Theorem for Boolean algebras]
\label{thm:stone2}
  For any Boolean algebra
  $\lparen B \mathcomma \vee \mathcomma \wedge \mathcomma \, \prime
  \mathcomma 0 \mathcomma 1 \rparen$
  there exists a totally disconnected compact Hausdorff space
  $\lparen X \mathcomma \mathcal{O} \rparen$,
  such that $B$ is isomorphic to the Boolean algebra of
  clopen\footnote{i.e.\ closed and open at the same time;
    cf.\ Example \ref{ex:clopen}}
  sets of $\lparen X \mathcomma \mathcal{O} \rparen$.
\end{theorem}

\begin{proof}
  We continue the proof of Theorem \ref{thm:stone1} and let
  $\symcal{U}$ be the set of ultrafilters on $B$
  as well as
  \[
    f
    \! \mathcolon
    B
    \rightarrow
    \Po{ \symcal{U} }
    \mathcomma \
    b
    \mapsto
    \lbrace
    U \in \symcal{U}
    \pipe
    b \in U
    \rbrace.
  \]
  Furthermore we denote by
  \[
    \symcal{B}
    \coloneq
    f \lbrack B \rbrack
    \equal
    \left \lbrace
    \lbrace
    U \in \symcal{U}
    \pipe
    b \in U
    \rbrace
    \pipe
    b \in B
    \right \rbrace
  \]
  the image of $f$.
  That $f$ is a monomorphism means
  that $\symcal{B}$ is the base
  of some topology~$\symcal{O}$ on $\symcal{U}$.
  Indeed, from $f \lparen 1 \rparen \equal \symcal{U}$
  one obtains $\bigcup \symcal{B} \equal \symcal{U}$,
  and from
  $f \lparen b \wedge c \rparen
    \equal f \lparen b \rparen \cap f \lparen c \rparen$
  we see that $B_1 \cap B_2 \in \symcal{B}$
  for any $B_1 \mathcomma B_2 \in \symcal{B}$.
  We now show that $\symcal{O}$ hast the desired properties.

  \smallskip

  \myul{$\symcal{O}$ is compact.} \thinspace
  Assume there is an open cover $\symcal{C}$ of $\symcal{U}$
  that has no finite subcover.
  Since~$\symcal{B}$ is a base for the topology,
  $\symcal{C}$ is of the form
  $\lbrace f \lparen i \rparen \pipe  i \in I \rbrace$
  for some $I \subseteq B$.
  We now fix a finite subset $I_0 \subseteq I$.
  By assumption
  $\bigcup_{ i \in I_0 } f \lparen i \rparen \ne \symcal{U}$,
  so there exists some ultrafilter~$U_{ I_0 } \in \symcal{U}$
  with $U_{ I_0 } \notin \bigcup_{ i \in I_0 } f \lparen i \rparen$.
  Thus, $U_{ I_0 } \notin f \lparen i \rparen$ for all $i \in I_0$
  which means nothing but $i \notin U_{ I_0 }$ for all $i \in I_0$.
  By Proposition \ref{prop:ultrafilter} we then have
  $i \prime \in U_{ I_0 }$ for all $i \in I_0$
  and since $I_0$ is finite also
  $\bigwedge_{ i \in I_0 } i \prime \in U_{ I_0 }$.
  In particular $\bigwedge_{ i \in I_0 } i \prime \ne 0$.
  Because $I_0$ was an arbitrary finite subset of $I$
  we have shown that the set
  $\lbrace i \prime \pipe i \in I \rbrace$
  fulfills the requirement of the corollary
  after Proposition \ref{prop:intersectionstable},
  so that there exists some ultrafilter~$U \in \symcal{U}$
  with~$\lbrace i \prime \pipe i \in I \rbrace \subseteq U$.
  But clearly $U \notin \bigcup_{ i \in I } f \lparen i \rparen$,
  contradicting the assumption that $\symcal{C}$ is an open cover.

  \smallskip

  \myul{$\symcal{O}$ is totally disconnected.} \thinspace
  Let $\symcal{V} \subseteq \symcal{U}$ contain
  at least two distinct elements $U \ne V$.
  Without loss of generality we find some $b \in U \setminus V$.
  Proposition \ref{prop:ultrafilter} states that $b \prime \in U$,
  hence
  $\varnothing \ne \lbrace W \in \symcal{V} \pipe b \in W \rbrace
  \subseteq f \lparen b \rparen$
  and
  $\varnothing \ne \lbrace W \in \symcal{V}
  \pipe b \prime \in W \rbrace \subseteq f \lparen b \prime \rparen$.
  Since no ultrafilter contains $b$ and $b \prime$ at the same time,
  one additionally obtains
  $f \lparen b \rparen \cap f \lparen b \prime \rparen
  \equal \varnothing$,
  whence $\symcal{V}$ is separated by disjoint open sets.

  \smallskip

  \myul{$\symcal{O}$ is Hausdorff.} \thinspace
  Let $U \mathcomma V \in \symcal{U}$ with $U \ne V$.
  As above there exists some $b \in B$ such that $b \in U$ and $b \prime \in V$.
  Apparently we have
  $U \in f \lparen b \rparen$ and $V \in f \lparen b \prime \rparen$,
  and furthermore again
  $f \lparen b \rparen \cap f \lparen b \prime \rparen \equal \varnothing$.

  \smallskip

  \myul{$\symcal{B}$ contains exactly the clopen sets.} \thinspace
  From
  $f \lparen b \rparen ^ \complement \equal f \lparen b \prime \rparen$
  for all $b \in B$ we see
  that every set in $\symcal{B}$ is both open and closed.
  Conversely, let $\symcal{V} \subseteq \symcal{U}$ be open and closed.
  Because $\symcal{B}$ is a base for the topology we have
  $\symcal{V} \equal \bigcup_{ i \in I } f \lparen i \rparen$
  for some $I \subseteq B$,
  and since  $\symcal{V}$ is a closed set of a compact topology
  there is some finite subset $I_0 \subseteq I$ with
  $\symcal{V} \equal \bigcup_{ i \in I_0 } f \lparen i \rparen
  \equal f \lparen \bigvee_{ i \in I_0} i \rparen$.
  In particular $\symcal{V} \in \symcal{B}$.
\end{proof}

\begin{definition}
  The topological space constructed
  in the proofs of Theorem \ref{thm:stone1} and Theorem \ref{thm:stone2}
  is called the \emph{Stone space} of $B$
  and denoted by $S \lparen B \rparen$.
\end{definition}

\medskip

Finally, we show a stronger version of Theorem \ref{thm:stone1}
for finite Boolean algebras.
For that reason we introduce the notion of \emph{atoms} on Boolean algebras
to gain better control of finite ultrafilters.

\medskip

\begin{definition}
  An element $a \in B$ of a Boolean algebra
  $\lparen B \mathcomma \vee \mathcomma \wedge \mathcomma \, \prime
  \mathcomma 0 \mathcomma 1 \rparen$
  with the properties
  \begin{enumerate}
    \item $a \ne 0$,
    \item $0 \leq b \leq a$ implies $b \in \lbrace 0 \mathcomma a \rbrace$
    for any $b \in B$
  \end{enumerate}
  is called an \emph{atom} of the Boolean algebra.
\end{definition}

\begin{proposition}
\label{prop:finite_ultrafilter}
  Every ultrafilter on a finite Boolean algebra
  $\lparen B \mathcomma \vee \mathcomma \wedge \mathcomma \, \prime
  \mathcomma 0 \mathcomma 1 \rparen$
  is of the form
  \[
    U_a
    \equal
    \lbrace b \in B \pipe a \leq b \rbrace
  \]
  for some atom $a \in B$.
\end{proposition}

\begin{proof}
  Let $U$ be an ultrafilter on a finite Boolean algebra.
  Note that also $U$ is finite,
  so the element $a \coloneq \bigwedge_{ u \in U } u \in U$
  is well-defined.
  Since clearly $a \leq u$ for all $u \in U$
  it follows that
  $U \subseteq \lbrace b \in B \pipe a \leq b \rbrace$,
  and the definition of filters also gives
  $\lbrace b \in B \pipe a \leq b \rbrace \subseteq U$.

  The filter $U$ is proper, so $a \ne 0$.
  Finally let $b \in B$ with $0 \leq b \less a$.
  Then $b \notin U$,
  hence by Proposition~\ref{prop:ultrafilter} it is $b \prime \in U$.
  This means $a \leq b \prime$ and therefore also $b \leq b \prime$.
  Lemma~\ref{la:order_complement} now shows that
  $0 \equal b \wedge b \equal b$,
  so that $a$ is an atom.
\end{proof}

\begin{lemma}
\label{la:atom_join}
  Let $a \in B$ be an atom of a Boolean algebra
  $\lparen B \mathcomma \vee \mathcomma \wedge \mathcomma \prime
  \mathcomma 0 \mathcomma 1 \rparen$
  and $b_1 \mathcomma \unicodeellipsis \mathcomma b_n \in B$ with
  $a \leq b_1 \vee \cdots \vee b_n$.
  Then $a \leq b_i$
  for some
  $i \in \lbrace 1 \mathcomma \unicodeellipsis \mathcomma n \rbrace$.
\end{lemma}

\begin{proof}
  Assume $a \nleq b_i$
  for all
  $i \in \lbrace 1 \mathcomma \unicodeellipsis \mathcomma n \rbrace$
  and fix some
  $i_0 \in \lbrace 1 \mathcomma \unicodeellipsis \mathcomma n \rbrace$.
  From the observation $0 \leq a \wedge b_{ i_0 } \prime \leq a$
  it follows that
  $a \wedge b_{ i_0 } \prime \in \lbrace 0 \mathcomma a \rbrace$,
  and because $a \nleq b_{ i_0 }$
  we obtain with Lemma \ref{la:order_complement}
  that $a \wedge b_{ i_0 } \prime \equal a$.
  The choice of $i_0$ was arbitrary,
  whence $a \leq b_i \prime$
  for
  all~$i \in \lbrace 1 \mathcomma \unicodeellipsis \mathcomma n \rbrace$
  and so
  \[
    a
    \leq
    b_1 \prime \wedge \cdots \wedge b_n \prime
    \equal
    \lparen b_1 \vee \cdots \vee b_n \rparen \prime.
  \]
  Therefore with the assumption $a \leq b_1 \vee \cdots \vee b_n$
  we finally obtain
  \[
    a
    \leq
    \lparen b_1 \vee \cdots \vee b_n \rparen
      \wedge \lparen b_1 \vee \cdots \vee b_n \rparen \prime
    \equal
    0,
  \]
  contradicting that $a$ is an atom.
\end{proof}

\begin{theorem}
  For any finite Boolean algebra
  $\lparen B \mathcomma \vee \mathcomma \wedge \mathcomma \, \prime
  \mathcomma 0 \mathcomma 1 \rparen$
  there exists a set $X$ such that $B$ is isomorphic $\Po{X}$.
\end{theorem}

\begin{proof}
  In the proof of Theorem \ref{thm:stone1} we have already seen that
  \[
    f
    \! \mathcolon
    B
    \rightarrow
    \Po{ \symcal{U} }
    \mathcomma \
    b
    \mapsto
    \lbrace
    U \in \symcal{U}
    \pipe
    b \in U
    \rbrace
  \]
  is a monomorphism from $B$
  into the powerset of the set $\symcal{U}$ of all ultrafilters on $B$.

  We now show that $f$ is surjective in case $B$ is finite.
  For this purpose let $\symcal{V} \in \Po{ \symcal{U} }$,
  and for $V \in \symcal{V}$ denote by $a_V$ the atom belonging to $V$
  according to Proposition \ref{prop:finite_ultrafilter}.
  Note that~$\symcal{V}$ is finite,
  so $a \coloneq \bigvee_{ V \in \symcal{V} } a_V \in B$ is well-defined.
  Let $U \in f \lparen a \rparen$ with corresponding atom $a_U$.
  Since $a \in V$ we have $a_U \leq a$,
  so by Lemma \ref{la:atom_join} it is $a_U \leq a_V$
  for some $V \in \symcal{V}$.
  As both $a_U$ and $a_V$ are atoms we obtain $a_U \equal a_V$
  and so $U \equal V$.
  This means $f \lparen a \rparen \subseteq \symcal{V}$,
  and since clearly also $\symcal{V} \subseteq f \lparen a \rparen$,
  we see that $f$ maps $a$ to $\symcal{V}$.
\end{proof}
