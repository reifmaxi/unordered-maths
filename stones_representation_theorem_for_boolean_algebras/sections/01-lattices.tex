\section{Lattices}
\label{sec:lattices}

In this section we develop the notion of \emph{Boolean algebras}
from the point of view of the theory of lattices.
To this end, we follow the first two chapters of the book
\textquote{Einführung in die Verbandstheorie} \cite{hermes:verbandstheorie-2}
and introduce successively bounded, distributive and complementary lattices,
to finally define Boolean algebras as lattices with all these properties.
In addition, we will see how each lattice gives rise to an order,
which lays the foundation for the representation theorems in
Section \ref{sec:representation_theorems}.

\medskip

\begin{definition}
\label{def:lattice}
  A \emph{lattice} is a tuple
  $\lparen L \mathcomma \vee \mathcomma \wedge \rparen$
  consisting of a set $L$ and two binary operations $\vee$ and $\wedge$ on $L$
  such that

  \begin{table}[h]
  \centering
    \begin{tabular}{rlcrl}
    Ass$_\vee$
    & $\lparen a \vee b \rparen \vee c \equal a \vee \lparen b \vee c \rparen$
    & \quad
    & Ass$_\wedge$
    & $\lparen a \wedge b \rparen \wedge c
      \equal a \wedge \lparen b \wedge c \rparen$ \\
    Com$_\vee$
    & $a \vee b \equal b \vee a$
    & \quad
    & Com$_\wedge$
    & $a \wedge b \equal b \wedge a$ \\
    Abs$_\vee$
    & $a \vee \lparen a \wedge b \rparen \equal a$
    & \quad
    & Abs$_\wedge$
    & $a \wedge \lparen a \vee b \rparen \equal a$ \\
    \end{tabular}
  \end{table}

  \noindent for all $a \mathcomma b \mathcomma c \in L$.
  The operation $\vee$ is called \emph{join} and $\wedge$ is called \emph{meet};
  the last two axioms are referred to as \emph{absorption laws}.
\end{definition}

\medskip

From the definition it is easy to see that the two lattice operations
are indistinguishable from the point of view of logic.
This observation invites to give proofs by means of a \emph{duality principle}
as follows:
We call a statement \emph{lattice-theoretic}
if besides the usual logical symbols only join and meet occur.
For a lattice-theoretic statement
one obtains the corresponding \emph{dual} statement
be exchanging join and meet.
Once a lattice-theoretic statement is proved,
one has thereby also proved its dual statement.
In fact, one obtains a proof for the latter
by exchanging join and meet in the original proof.
The justification for each of the new conclusions
is given by the respective dual axiom of Definition \ref{def:lattice}.
A simple application of this duality principle can be found
in the proof of the following proposition.

Note that what has been stated above
is not a theorem within the theory of lattices,
but rather a meta theorem \emph{about} this theory.

\medskip

\begin{proposition}
  In a lattice
  $\lparen L \mathcomma \vee \mathcomma \wedge \rparen$
  the join and the meet are idempotent,
  that is we have
  $a \vee a \equal a$
  and
  $a \wedge a \equal a$
  for all
  $a \in L$.
\end{proposition}

\begin{proof}
  Let $a \in L$.
  Then
  \[
    a \vee a
    \equal
    a \vee \lparen a \wedge \lparen a \vee a \rparen \rparen
    \equal
    a
  \]
  and
  \[
    a \wedge a
    \equal
    a \wedge \lparen a \vee \lparen a \wedge a \rparen \rparen
    \equal
    a,
  \]
  where we used Abs$_{\wedge}$ and Abs$_{\vee}$ for the first equation
  and the corresponding duals in the latter ones.
\end{proof}

\begin{lemma}
  Let
  $\lparen L \mathcomma \vee \mathcomma \wedge \rparen$
  be a lattice.
  It is
  \[
    a \vee b \equal b
    \Leftrightarrow
    a \wedge b \equal a
  \]
  for all $a \mathcomma b \in L$.
\end{lemma}

\begin{proof}
  Let $a \mathcomma b \in L$.

  \fbox{$\Rightarrow$}
  We have
  \[
    a
    \equal
    a \wedge \lparen a \vee b \rparen
    \equal
    a \wedge b,
  \]
  where we used Abs$_{\wedge}$ for the first identity
  and the assumption for the second one.

  \fbox{$\Rightarrow$}
  Note that we showed the first implication for arbitrary elements of $L$,
  hence we are free to interchange $a$ and $b$.
  Consequently we already proved
  \[
    b \vee a \equal a
    \Rightarrow
    b \wedge a \equal b,
  \]
  which is equivalent to
  \[
    a \vee b \equal a
    \Rightarrow
    a \wedge b \equal b
  \]
  due to the commutativity of the join and the meet.
  Since this latter assertion is exactly
  the dual statement of the remaining implication,
  we are done by the duality principle.
\end{proof}

\medskip

Even if the preceding result seems somewhat unimpressive at first glance,
it has far-reaching consequences.
Proposition \ref{prop:lattice_order} shows how an order relation arises
from the proven equivalence,
which plays an important role in Section \ref{sec:representation_theorems}.

\medskip

\begin{proposition}
\label{prop:lattice_order}
  Let
  $\lparen L \mathcomma \vee \mathcomma \wedge \rparen$
  be a lattice.
  Then
  \[
    a \leq b
    \mathcolon \Leftrightarrow
    a \vee b \equal b
    \Leftrightarrow
    a \wedge b \equal a
  \]
  for
  $a \mathcomma b \in L$
  is a partial order on $L$
  with respect to which for every two elements their infimum and supremum exist.
\end{proposition}

\begin{proof}
  Let $a \mathcomma b \mathcomma c \mathcomma d \in L$.
  We first show that $\leq$ is a partial order on $L$.
  Reflexivity is given by the idempotence of the join and the meet.
  If $a \leq b$ and $b \leq c$,
  that is $a \vee b \equal b$ and $b \vee c \equal c$,
  we have
  \[
    a \vee c
    \equal
    a \vee \lparen b \vee c \rparen
    \equal
    \lparen a \vee b \rparen \vee c
    \equal
    b \vee c
    \equal
    c,
  \]
  so that transitivity of $\leq$
  follows from the associativity of the lattice operations.
  Similarly, antisymmetry is obtained by
  the commutativity of the join and the meet.

  Since
  \[
    \lparen a \wedge b \rparen \vee a
    \equal
    a \vee \lparen a \wedge b \rparen
    \equal
    a
    \quad \text{and} \quad
    \lparen a \wedge b \rparen \vee a
    \equal
    b \vee \lparen b \wedge a \rparen
    \equal
    b
  \]
  we see  by the first of the two equivalent definitions of $\leq$
  that $a \wedge b$ is a lower bound of the set
  $\lbrace a \mathcomma b \rbrace$.
  If $d$ is another lower bound of
  $\lbrace a \mathcomma b \rbrace$
  we have
  \[
    d \wedge \lparen a \wedge b \rparen
    \equal
    \lparen d \wedge d \rparen \wedge \lparen a \wedge b \rparen
    \equal
    \lparen d \wedge a \rparen \wedge \lparen d \wedge b \rparen
    \equal
    d \wedge d
    \equal
    d,
  \]
  where we used the second definition of $\leq$
  as well as the idempotence of the meet.
  The latter observation states that $d \leq a \wedge b$,
  so $a \wedge b$ is the greatest lower bound of
  $\lbrace a \mathcomma b \rbrace$.

  Dually it is $a \vee b$ the supremum of
  $\lbrace a \mathcomma b \rbrace$.
  To show this one uses in each step
  the respective other definition of the order
  and as usual the corresponding dual axiom.
\end{proof}

\medskip

The next result is the converse of Proposition \ref{prop:lattice_order}.
In fact, one easily obtains that both constructions
are inverse to each other in the following sense:
For a lattice,
applying Proposition \ref{prop:order_lattice} to the arising order
yields the original lattice again,
and vice versa.
Thus, the lattices correspond exactly
to ordered sets in which infima and suprema exist for all finite subsets.

\medskip

\begin{proposition}
\label{prop:order_lattice}
  Let
  $\lparen X \mathcomma \leq \rparen$
  be a partially ordered set
  such that for every two elements their infimum and supremum exists.
  Then
  $\lparen X \mathcomma \vee \mathcomma \wedge \rparen$
  is a lattice,
  where
  \[
    x \vee y
    \coloneq
    \sup \lbrace x \mathcomma y \rbrace
    \quad \text{and} \quad
    x \wedge y
    \coloneq
    \inf \lbrace x \mathcomma y \rbrace.
  \]
\end{proposition}

\begin{proof}
  Let $x \mathcomma y \mathcomma z \in X$.
  Clearly
  $\sup \lbrace \sup \lbrace x \mathcomma y \rbrace \mathcomma z \rbrace
  \equal \sup \lbrace x \mathcomma y \mathcomma z \rbrace
  \equal \sup \lbrace x \mathcomma \sup \lbrace y \mathcomma z \rbrace \rbrace$
  as well as
  $\sup \lbrace x \mathcomma y \rbrace
  \equal \sup \lbrace y \mathcomma x \rbrace$,
  so that $\vee$ is associative and commutative.
  Furthermore one easily sees that
  $\sup \lbrace x \mathcomma \inf \lbrace x \mathcomma y \rbrace \rbrace
  \equal x$,
  hence $\vee$ also fulfills Abs$_{\vee}$.
  In the same manner, the desired properties of $\wedge$ are obtained.
\end{proof}

\medskip

We now begin to successively introduce new kind of lattices
and point out some of the consequences that immediately follow.
In this way, we gain a better understanding
of the properties required in the definition of Boolean algebras.

\medskip

\begin{definition}
  A lattice
  $\lparen L \mathcomma \vee \mathcomma \wedge \rparen$
  is called \emph{bounded},
  if there exist a least and a greatest element of $L$
  with respect to the induced order from Proposition \ref{prop:lattice_order}.
  In this case these elements are unique
  (cf.\ the remark after Definition \ref{def:ordered_elements})
  and we denote the least element by $0$
  and the greatest element by $1$.
\end{definition}

\begin{remark}
  One immediately sees from the definition of the order that
  in a bounded lattice
  $\lparen L \mathcomma \vee \mathcomma \wedge \rparen$
  the identities
  \[
    0 \vee a \equal a,
    \quad
    0 \wedge a \equal 0,
    \quad
    a \vee 1 \equal 1,
    \quad
    a \wedge 1 \equal a
  \]
  hold for all $a \in L$.
  In particular,
  the elements $0$ and $1$ are uniquely characterized
  as neutral elements of the join and the meet, respectively.
\end{remark}

\begin{definition}
  Let
  $\lparen L \mathcomma \vee \mathcomma \wedge \rparen$
  be a bounded lattice and $a \mathcomma b \in L$.
  We call $b$ a \emph{complement} of $a$ if
  \[
    a \vee b \equal 1
    \quad \text{and} \quad
    a \wedge b \equal 0.
  \]
  In this case also $a$ is a complement of $b$
  and we say that $a$ and $b$ are \emph{complementary}.
  The lattice itself is called \emph{complementary}
  if every element has at least one complement.
\end{definition}

\begin{remark}
  In every bounded lattice $0$ and $1$ are complementary.
\end{remark}

\newpage

\begin{example}
  Let $\mathcal{S}$ be the set of all subspaces of the $\Q$-vector space $\Q^2$.
  By basic linear algebra we know that
  \[
    U \vee V
    \coloneq
    U \mathplus V
    \coloneq
    \lbrace
      u \mathplus v
    \pipe
      u \in U \text{ and } v \in V
    \rbrace
  \]
  and
  \[
    U \wedge V
    \coloneq
    U \cap V
  \]
  are well-defined operations on $\mathcal{S}$ that make
  $\lparen \mathcal{S} \mathcomma \vee \mathcomma \wedge \rparen$
  a bounded lattice
  with least element~$\lbrace 0 \rbrace$
  and greatest element $\Q^2$.
  Furthermore, the Steinitz exchange lemma
  ensures the existence of complements.
  Note that these are not unique in general, as
  \[
  \mathrm{span}
  \begin{pmatrix}
    1 \\
    0
  \end{pmatrix}
  \mathcomma
  \quad
  \mathrm{span}
  \begin{pmatrix}
    0 \\
    1
  \end{pmatrix}
  \mathcomma
  \quad
  \mathrm{span}
  \begin{pmatrix}
    1 \\
    1
  \end{pmatrix}
  \]
  are all pairwise complementary.
\end{example}

\begin{remark}
  For any ring $R$ and any left $R$-module $M$
  the previous construction makes the set of $R$-submodules of $M$
  a bounded lattice.
  However, the property of being complementary may not be fulfilled in general.
  For example, let $R$ be a non-trivial local ring,
  that is a unitary ring with more than two elements
  and a unique maximal left ideal $\mathfrak{m}$.
  If $R$ is considered as left module over itself,
  the corresponding submodules are exactly the left ideals of $R$.
  In this situation $\mathfrak{m}$ has no complement,
  since it is maximal and not the zero ideal.
\end{remark}

\begin{definition}
  A lattice
  $\lparen L \mathcomma \vee \mathcomma \wedge \rparen$
  is called \emph{distributive} if
  \[
    a \vee \lparen b \wedge c \rparen
    \equal
    \lparen a \vee b \rparen  \wedge \lparen a \vee c \rparen
    \quad \text{and} \quad
    a \wedge \lparen b \vee c \rparen
    \equal
    \lparen a \wedge b \rparen \vee \lparen a \wedge c \rparen
  \]
  for all $a \mathcomma b \mathcomma c \in L$.
\end{definition}

\begin{proposition}
\label{prop:complements}
  Let
  $\lparen L \mathcomma \vee \mathcomma \wedge \rparen$
  be a bounded and distributive lattice.
  Then every $a \in L$ has at most one complement.
\end{proposition}

\begin{proof}
  Let $a \mathcomma b_1 \mathcomma b_2 \in L$.
  If $b_1$ and $b_2$ are complements of $a$ we have
  $b_1 \vee a \equal b_2 \vee a$
  as well as
  $b_1 \wedge a \equal b_2 \wedge a$.
  Hence,
  \begin{align*}
    b_1
    \equal
    b_1 \vee \lparen b_1 \wedge a \rparen
    &\equal
    b_1 \vee \lparen b_2 \wedge a \rparen
    \equal
    \lparen b_1 \vee b_2 \rparen \wedge \lparen b_1 \vee a \rparen \\
    &\equal
    \lparen b_2 \vee b_1 \rparen \wedge \lparen b_2 \vee a \rparen
    \equal
    b_2 \vee \lparen b_1 \vee a \rparen
    \equal
    b_2 \vee \lparen b_2 \wedge a \rparen
    \equal
    b_2.
  \end{align*}
\end{proof}

\begin{definition}
  Let
  $\lparen L \mathcomma \vee \mathcomma \wedge \rparen$
  be a bounded and distributive lattice.
  If $a \in L$ has a complement
  we denote it by $a \prime$.
\end{definition}

\begin{corollary}
  In a complementary distributive lattice
  $\lparen L \mathcomma \vee \mathcomma \wedge \rparen$
  it is
  $a \equal \lparen a \prime \rparen \prime$
  for all $a \in L$.
\end{corollary}

\begin{proof}
  Both $a$ and $\lparen a \prime \rparen \prime$
  are complements of $a \prime$.
  Now use Proposition~\ref{prop:complements}.
\end{proof}

\begin{proposition}
  Let
  $\lparen L \mathcomma \vee \mathcomma \wedge \rparen$
  be a complementary distributive lattice.
  Then
  \[
    \lparen a \vee b \rparen \prime
    \equal
    a \prime \wedge b \prime
    \quad \text{and} \quad
    \lparen a \wedge b \rparen \prime
    \equal
    a \prime \vee b \prime
  \]
  for all
  $a \mathcomma b \in L$.
\end{proposition}

\begin{proof}
  Let $a \mathcomma b \in L$.
  We have
  \[
    \lparen a \vee b \rparen \wedge \lparen a \prime \wedge b \prime \rparen
    \equal
    \lparen a \wedge a \prime \wedge b \prime \rparen
      \vee \lparen b \wedge a \prime \wedge b \prime \rparen
    \equal
    0 \vee 0
    \equal
    0
  \]
  and
  \[
    \lparen a \vee b \rparen \vee \lparen a \prime \wedge b \prime \rparen
    \equal
    \lparen a \vee b \vee a \prime \rparen
      \wedge \lparen a \vee b \vee b \prime \rparen
    \equal
    1 \wedge 1
    \equal
    1,
  \]
  so that
  $a \prime \wedge b \prime$
  is the (unique) complement of
  $a \vee b$.
  The second identity is obtained analogously.
\end{proof}

\begin{definition}
  A complementary distributive lattice is called a \emph{Boolean algebra}.
  We say a Boolean algebra is \emph{degenerate}
  in case it consists of a single element only.
\end{definition}

\begin{remark}
  By definition Boolean algebras are bounded lattices and hence nonempty.
  Furthermore, in any non-degenerate Boolean algebra it is $0 \ne 1$.
\end{remark}
