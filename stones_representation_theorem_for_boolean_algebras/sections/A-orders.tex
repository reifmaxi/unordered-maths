\section{Partially ordered sets}
\label{sec:orders}

This section briefly collects some basic notions of order theory
which occur within the study of Boolean algebras.
The only theorem in this section (Zorn's lemma) remains without proof,
but it is well-known to be equivalent to the axiom of choice.

\medskip

\begin{definition}
  A \emph{(partially) ordered set} is a tuple
  $\lparen X \mathcomma \leq \rparen$
  consisting of a set $X$ and a relation $\leq$ on $X$ such that
  \begin{enumerate}
    \item $x \leq x$ for all $x \in X$,
    \item $x \leq y$ and $y \leq z$ implies $x \leq z$
    for all $x \mathcomma y \mathcomma z \in X$,
    \item $x \leq y$ and $y \leq z$ implies $x \equal y$
    for all $x \mathcomma y \in X$.
  \end{enumerate}
  In this case we say that $\leq$ is a (partial) order on $X$.
\end{definition}

\begin{definition}
\label{def:ordered_elements}
  Let
  $\lparen X \mathcomma \leq \rparen$
  be a partially ordered set and
  $S \subseteq X$.
  An element $x \in X$ is called
  \begin{itemize}
    \item a \emph{lower bound} of $S$ if
    $x \leq s$ for all $s \in S$,
    \item an \emph{upper bound} of $S$ if
    $s \leq x$ for all $s \in S$,
    \item a \emph{least element} of $S$ if
    $x$ is a lower bound of $S$ and $x \in S$,
    \item a \emph{greatest element} of $S$ if
    $x$ is an upper bound of $S$ and $x \in S$,
    \item an \emph{infimum} of $S$ if
    $x$ is a greatest element in the set of lower bounds of $S$,
    \item a \emph{supremum} of $S$ if
    $x$ is a least element in the set of upper bounds of $S$,
    \item a \emph{minimal element} of $S$ if
    $x \in S$, and $s \leq x$ implies $s \equal x$ for all $s \in S$,
    \item a \emph{maximal element} of $S$ if
    $x \in S$, and $x \leq s$ implies $s \equal x$ for all $s \in S$.
  \end{itemize}

\begin{remark}
  If existing,
  a least or a greatest element of $S$ is unique due to antisymmetry.
  This is of course also true for a infimum or a supremum of $S$,
  so that in case of existence we may use the notations
  $\inf \lparen S \rparen$ or $\sup \lparen S \rparen$, respectively.
\end{remark}

\begin{definition}
  Let $\lparen P \mathcomma \leq \rparen$ be a partially ordered set.
  A subset $C \subseteq P$ is called a \emph{chain},
  if always $c \leq d$ or $d \leq c$ for any $c \mathcomma d \in C$.
\end{definition}

\begin{theorem}[Zorn's lemma]
\label{thm:zorn}
  Let $\lparen P \mathcomma \leq \rparen$ be a partially ordered set
  such that every chain in $P$ has an upper bound.
  Then there exists a maximal element in $P$.
\end{theorem}

\end{definition}

\begin{definition}
  A \emph{filter} on a partially ordered set
  $\lparen X \mathcomma \leq \rparen$
  is a subset
  $F \subseteq X$
  such that
  \begin{enumerate}
    \item $F \ne \varnothing$,
    \item $f \leq x$ implies $x \in F$
    for all $f \in F$ and $x \in X$,
    \item there is $h \in F$ with $h \leq f$ and $h \leq g$
    for all $f \mathcomma g \in F$.
  \end{enumerate}
  A filter is called \emph{proper} in case $F \ne X$.
\end{definition}

\begin{definition}
  The set of proper filters on $X$ is partially ordered by set inclusion.
  A maximal element in this set is called an \emph{ultrafilter}.
\end{definition}

\begin{remark}
  Not every partially ordered set has an ultrafilter.
  Consider the integers $\Z$ with the usual order.
  One immediately sees that the proper filters are exactly the segments,
  that is sets of the form
  $\lbrace x \in \Z \pipe z \leq x \rbrace$
  for a given $z \in \Z$.
  Thus,
  since $\Z$ is unbounded downwards,
  there no proper filter that is maximal with respect to the set inclusion.
\end{remark}
